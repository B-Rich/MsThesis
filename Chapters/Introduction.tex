% Chapter 0

\chapter{Introduction} % Main chapter title

\label{Chapter1} % For referencing the chapter elsewhere, use \ref{Chapter1} 

\lhead{Chapter 1. \emph{Introduction}} % This is for the header on each page - perhaps a shortened title

In an increasing world population, energy consumption is directly linked to the survival of the human specie. 
With three quarter of the world energy source coming from fossil fuel in 2004 \cite{Beretta}, the world is subjected to an environmental crisis linked to our energy sources. The need for clean and renewable energy has never been important as it is today.
Geothermal energy classified as clean and sustainable energy has the potential of contributing significantly to sustainable energy use in many parts of the world \cite{Ax2005}. Generated by radioactive elements in the crust and lava within the earth, geothermal energy provides an unlimited source of energy for the world population. \\

Geothermal energy production at a sustainable rate can be maintained for $100-300$ years \cite{Ax2005, Axelsson2008}. This requires  knowledge and understanding of the geothermal system, sustainable production schemes such as reinjection of cold water and good management strategies. These strategies involve a multidisciplinary field of studies comprising geology, geophysics, chemistry and mathematics. With the development of modern computer architectures, numerical simulation combined with geological, geophysical and chemical data, play a key role in geothermal resource management.\\

Many pioneering works have been done in the field of geothermal engineering. Axelsson, Barker and Stefansson presented reinjection of water as a resource management strategy 
\cite{Ax2005, Axelsson2005, Ax-R08, Barker1995, Stefansson1997}. A method for analysing tracer test data used in reinjection was derived by Axelsson et. al \cite{Axelsson2005}. A lumped parameter modelling technique  was developed by Axelsson to simulate pressure response due to production in a low temperature geothermal system 
\cite{Ax-Rel-Lump05, Ax-Simul-Lump89}. Bjornsson developed the well bore simulator HOLA capable of simulating steady state two phase flow in a multi feed zones geothermal well \cite{Bjornsson1987}. Bodvarson was the first to derive an analytical method for calculating the thermal front velocity induced during injection of cold water in a hot geothermal reservoir for constant rock and fluid properties \cite{Bod-R72}. His method was latter extended for temperature independent rock and fluid properties by Stoppa et al and Achou \cite{Achou2013, Waj05}. \\

The main objective of this study is to present different methods used in modelling geothermal systems, in particular low temperature geothermal systems. volumetric and dynamic methods are presented. Reinjection is presented as an essential part of sustainable production  of geothermal energy. The lumped parameter modelling method is applied to a low temperature geothermal reservoir located in Munadarnes in west Iceland. The simulation reveals that the reservoir permeability is high. Multiple reinjection scenarios with different production and reinjection rate show that reinjection increases significantly the pressure of the reservoir. By using an analytical solution of water flow in a one dimension channel, the cooling response of the production well due to injection of cold water is presented. Cooling is minimised by placing the reinjection well at a few $km$ from the production well. By revisiting the work of Stoppa et. al \cite{Waj05}, an analytical method based on the theory of conservation laws is used to derived the thermal front velocity induced by reinjection.\\

The thesis is organised as follows: In chapter 2, properties of hydrothermal and geothermal systems are discussed. Hydrological and thermal properties of hydrothermal systems are presented. Geothermal systems are classified and methods for geothermal resource assessment are presented. In chapter 3, the thermal front velocity of cold water is derived using an analytical method.
 Chapter 4 present a case study using lumped parameters modelling. Chapter 5 contains conclusions and discussion and the possible extension or improvement of this work. 




%%----------------------------------------------------------------------------------------



%----------------------------------------------------------------------------------------
