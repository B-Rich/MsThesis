% Chapter 

\chapter{Conclusion and future work} % Main chapter title

\label{Chapter5} % For referencing the chapter elsewhere, use \ref{Chapter1} 

\lhead{Chapter 5. \emph{Conclusion and future work}} % This is for the header on each page - perhaps a shortened title

%----------------------------------------------------------------------------------------


%\section{ Discussions, conclusion and future work}
The focus of this thesis was to model a low temperature geothermal system by using two dynamical methods:  A lumped parameter modelling for simulating pressure change and an analytical method for determining the velocity of cold water movement during injection. A data set was provided by Reykjavik Energy for lumped parameter simulation of well $MN 8$. The sample data consisted of water level (Pressure) and production rate from 2003 to 2008. This data appears to to be spread out, possibly due to measurement error. Using a two tanks open model the simulation was performed with a coefficient of determination of about 56 $\%$. From the lumped parameter modelling of well $MN 8$ located in Munadarnes in west Iceland, the Munadarnes reservoir permeability  was estimated to be about 68.8 $mD$. Most geothermal system permeability is in the range of $1-100$ $mD$.
The reservoir size was estimated to be about 10.5 $km^{2}$, with unconfined recharge mechanism. \\

Based on simulation parameters from the lumped parameters modelling, a 20 years production scenario without and with reinjection revealed that reinjecrion mitigates pressure drown down due to production. For a current production rate of $8.2$ $l/s$ and a reinjection water of $20 \,^{\circ}{\rm c}$ a  scenario without injection shows that over 20 years the water level will drop by 7.4 $m$. However with reinjection at a rate of 5 $l/s$ the water level with decrease by 2.4 $m$. Renjecting cold water into a hot reservoir can cause the reservoir rock to cool down. To mitigate this cooling effect the injection well must be placed within the production zone and at a few $km$ from the production well. From a theoretical solution of fluid flow in a one dimensional channel the cooling of well $MN 08$ was studied over 10 years. The cooling of well $MN 08$ depends on three factors: The injection rate $q$, the production rate $Q$ and the distance between the production well and the injection well.  The injection rate was set at 5 $l/s$ while the production rate varies from $7.2$ to $10.2$ $l/s$. From an initial reservoir temperature of $86 \,^{\circ}{\rm c}$ the temperature of the reservoir over 10 years was about $58\,^{\circ}{\rm c}$ for a production rate of $10.2$ $l/s$ and a distance of 500 $m$ between production well and injection well. When the distance between production and injection well was increased to 2 $km$ the reservoir temperature over 10 years was about $71 \,^{\circ}{\rm c}$.\\
For the same injection rate of 5 $l/s$ and production rate of 7.2 $l/s$ the temperature of the well was $46 \,^{\circ}{\rm c}$ and $64 \,^{\circ}{\rm c}$ for a distance of 500 $m$ and 2 $km$ respectively. This results shows that to minimise the cooling effect of reinjection the injection well must be place at few $km$ from the production well within the production zone. increase in production rate also mitigate cooling during injection of cold water. Accurate position of injection well must however be determined after a tracer test experiment.\\
\\
The velocity of the cold injected water was also derived from the theory of hyperbolic conservations laws. The result present in this  work was compared with the one obtained by Stoppa et al. A relative error of $10^{-3}$ was observed between the two results. The numerical evaluation of the analytical expressions was derive using a simple numerical integration scheme: The trapezoidal rule.\\

This work can naturally be extended to include a lumped parameter modelling of hight temperature reservoirs. In this case the two phase flow nature of high temperature wells must be included. The injection of cold water in two phase flow reservoirs can also be studied. Due to the complexity of two phase flow a numerical method might be necessary. 

%----------------------------------------------------------------------------------------
